\documentclass[a4paper,10pt]{article} % type, taille police

\usepackage[utf8]{inputenc} % encodage
\usepackage[T1]{fontenc} % encodage
\usepackage[french]{babel} % gestion du français
\usepackage{amssymb} % symboles mathématiques
\usepackage{textcomp} % flèche,  intervalle
\usepackage{stmaryrd} % intervalle entiers

\usepackage[left=3cm,right=3cm,top=3cm,bottom=3cm]{geometry} % marges
\usepackage[hidelinks]{hyperref} % sommaire interactif dans un pdf
\usepackage[nottoc, notlof, notlot]{tocbibind} % affichage des références dans la table des matières (?)
\usepackage{float} % placement des figures
\usepackage[toc,page]{appendix} % ajout d'annexes
\usepackage{amsthm} % format des déf, prop...
\usepackage{amsmath} % matrices, ...
\usepackage{multirow} % fusionner cellules verticalement

%\usepackage[french,ruled,vlined,linesnumbered]{algorithm2e} % affichage d'algorithmes
\usepackage{tikz} % affichage de schémas
\usepackage{graphicx} % affichage d'images
\usepackage{url} % inclure des urls
\usepackage{bbold} % fonction caractéristique 1

\renewcommand{\appendixtocname}{Annexes} % renommage annexes
\renewcommand{\appendixpagename}{Annexes}

%\providecommand{\SetAlgoLined}{\SetLine} % paramètre pour algorithm2e
%\providecommand{\DontPrintSemicolon}{\dontprintsemicolon}  % paramètre pour algorithm2e

\definecolor{bgreen}{rgb}{0.30,0.70,0}

\theoremstyle{definition} % pas d'italique pour le format des déf, prop...
\newtheorem{thm}{Théorème} % \begin{thm} \end{thm}
\newtheorem{cor}[thm]{Corollaire}
\newtheorem{defi}[thm]{Définition}
\newtheorem{ex}[thm]{Exemple}
\newtheorem{lem}[thm]{Lemme}
\newtheorem{rem}[thm]{Remarque}
\newtheorem{conj}[thm]{Conjecture}

\newcommand{\R}{\mathbb{R}}
\newcommand{\Q}{\mathbb{Q}}
\newcommand{\N}{\mathbb{N}}
\newcommand{\Z}{\mathbb{Z}}
\newcommand{\inorm}[2]{{\left\lVert{#1}\right\rVert_{#2}}}
\newcommand{\infnorm}[1]{\inorm{#1}{}}
\newcommand{\clength}[1]{\operatorname{length}(#1)}
\newcommand{\twonorm}[1]{\inorm{#1}{2}}
\newcommand{\gpacrec}{R_{GPAC}}
\newcommand{\gpacpoly}{P_{GPAC}}
\newcommand{\gpacnpoly}{NP_{GPAC}}
\newcommand{\gpacspace}{PSPACE_{GPAC}}

%#############################################################################################################%
%#############################################################################################################%
%#############################################################################################################%

\title{Reconnaissance et indexation de formes}
\author{Quentin Cormier \and Yassine Hamoudi}
\date{4 mai 2015}

\begin{document}

\maketitle

\tableofcontents

%#############################################################################################################%
%#############################################################################################################%
%#############################################################################################################%

\section{Introduction}

%#############################################################################################################%
%#############################################################################################################%
%#############################################################################################################%

\section{Méthode}


%#############################################################################################################%
%#############################################################################################################%
%#############################################################################################################%

\section{Résultas}

%-------------------------------------------------------------------------------------------------------------%

  \subsection{Sensibilité aux perturbations}

Les valeurs propres du Laplacien de Dirichlet vérifient un certain nombre de propriétés mathématiques qui garantissent que notre descripteur est insensible au redimensionnement, à la rotation et à la translation. Nous vérifions expérimentalement ces propriétés ci-dessous.

\underline{Redimensionnement} Etant donné un domaine $\Omega$ et un facteur $a > 0$, on a $\lambda_k(a \Omega) = \frac{\lambda_k(\Omega)}{a^2}$. Or, notre descripteur utilise des rapports de valeurs propres, il est donc inchangé par redimensionnement : $\frac{\lambda_k(a \Omega)}{\lambda_m(a \Omega)} = \frac{\lambda_k(\Omega)}{\lambda_m(\Omega)}$.



%#############################################################################################################%
%#############################################################################################################%
%#############################################################################################################%

\section{Discussion}

%#############################################################################################################%
%#############################################################################################################%
%#############################################################################################################%

\section{Conclusion}

%#############################################################################################################%
%#############################################################################################################%
%#############################################################################################################%

\section*{Bonus}

Nous avons essayé de reconstruire un son à partir des valeurs propres du Laplacien de Dirichlet. Pour cela, {\large \textcolor{red}{explication du procédé utilisé}}

Afin d'entendre le son associé à l'image \texttt{beetle-11.pgm} par exemple, entrer : 
\begin{center}
  \texttt{python3 sound.py database/beetle-11.pgm}
\end{center}

Nous avons également développé un petit jeu, accessible par :
\begin{center}
  \texttt{python3 sound\_game.py}
\end{center}


Il s'agit de retrouver parmi les sons de plusieurs objets celui appartenant à la même catégorie qu'un motif de départ. 

%#############################################################################################################%
%#############################################################################################################%
%#############################################################################################################%

\bibliographystyle{alpha}
\bibliography{Biblio}
\nocite{*}

\end{document}